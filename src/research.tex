% mainfile: cv.tex
% \subsection*{Research Interests}
% \begin{itemize}
  % \item
	% \ressubheading{DOE Joint Genome Institute (Lawrence Berkeley National Lab)}{Walnut Creek, CA}{Researcher in Analysis Group under Dr. Zhong Wang}{Summer 2010}
	% \begin{itemize}
	    % \resitem{Created \textbf{open source} genome validation software tool in \texttt{python} and \texttt{C}}
            % \resitem{Used machine learning to mine TBs of genome data efficiently using novel likelihood function}
	% \end{itemize}
% \end{itemize}
% \subsection*{Research Interests}
\par{\textbf{Research Focus: } Text Mining, Knowledge Discovery from Text, Deep Learning for NLP (word2vec)\vspace{0.5em}
\begin{itemize}
  \item \textit{Application area:} \textbf{Text Analytics Methods for Public Health Surveillance}\\
\end{itemize}
        \textbf{Broad Focus:} Data Science, Machine Learning and Pattern Recognition.\vspace{0.5em}
}
\vspace{1em}
\par{\textbf{Research Problems:} Mining online news media for modeling spread of infectious diseases\vspace{0.5em}
\begin{itemize}
 \item  \textbf{Forecasting rare disease outbreaks from multiple news sources}
   \begin{itemize}
     \item {\textbf{SourceSeer}}: Real-time forecasting of rare disease outbreaks in multiple countries of Latin America 
       using spatio-temporal topic modeling over online health-related news articles provided by HealthMap.
     \item \textit{Designed and implemented the entire big-data pipeline for sending hantavirus forecasts to IARPA in real-time}  
     \item \textit{Projects:} EMBERS for IARPA OSI Program (winning team)
   \end{itemize}\vspace{0.25em}
 \item \textbf{Assessing temporal associations between news trends and infectious disease outbreaks}
   \begin{itemize}
     \item {\textbf{EpiNews}}: Designed and implemented a supervised temporal topic 
       model for quantifying media coverage during infectious disease outbreaks.
     \item \textit{Projects:} EMBERS for IARPA OSI Program.
   \end{itemize}\vspace{0.25em}
 \item \textbf{Neural word embeddings for characterizing infectious diseases}
   \begin{itemize}
     \item Designed and implemented a vocabulary driven word2vec method for modeling infectious diseases and constituent attributes (such as symptoms)
           from online health-related news articles provided by HealthMap.
   \end{itemize}
 \item \textbf{Dependency parsing based shortest distance and negation detection approaches guided by neural word embeddings for automated construction of epidemiological line lists}
   \begin{itemize}
     \item {\textbf{Guided Deep List}}: Designed and implemented the first tool for building automated line lists (in near real-time) from
       WHO Disease Outbreak News (DONs) with specific focus on emerging diseases.
     \item \textit{Projects:} EMBERS for IARPA OSI Program.
   \end{itemize}
\end{itemize}
}
